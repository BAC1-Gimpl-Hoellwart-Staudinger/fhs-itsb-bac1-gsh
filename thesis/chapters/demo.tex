\section{Demo}
\renewcommand{\pageAuthor}{Clemens Gimpl}
\label{sec:theorie}

Eine Formel in \LaTeX\  schreibt man so:

\begin{align}
	r_k = \sum_{n=0}^{N-1} x_n e^{-2\pi j\frac{nk}{N}}
	\label{eq:fourier}
\end{align}

Bei Herleitungen nummeriert man aber vielleicht nicht jede Zeile, so wie
bei \cref{eq:fourier}:
\begin{align*}
	1 + 2 + \dots + n &= \sum_{k=1}^n k \\
	&= \frac{n \cdot (n+1)}{2}
\end{align*}

Es geht aber auch so. Dabei keine Einrückungen machen, weil die so gedruckt werden wie sie im *.tex file sind.

Beispielzitat \cite{_DINISO924111_}
